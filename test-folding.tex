\documentclass{article}
\usepackage{amsmath}
\usepackage{graphicx}

\title{Test Document for Code Folding}
\author{CollabTeX Team}
\date{\today}

\begin{document}

\maketitle

\section{Introduction}
This is the introduction section. It contains some text that should be foldable
when you click on the fold indicator in the gutter next to the section command.

\subsection{Background}
This subsection should also be foldable independently. When folded, only the
subsection title should be visible.

\section{Methods}
Here's another section that can be folded.

\begin{itemize}
    \item First item
    \item Second item
    \item Third item with some longer text that spans
          multiple lines to test folding
\end{itemize}

\begin{equation}
    E = mc^2
\end{equation}

\section{Results}
The results section contains various environments:

\begin{figure}[h]
    \centering
    % \includegraphics[width=0.5\textwidth]{example.png}
    \caption{Example figure that should be foldable}
    \label{fig:example}
\end{figure}

\begin{table}[h]
    \centering
    \begin{tabular}{|c|c|c|}
        \hline
        Column 1 & Column 2 & Column 3 \\
        \hline
        Data 1 & Data 2 & Data 3 \\
        \hline
    \end{tabular}
    \caption{Example table}
    \label{tab:example}
\end{table}

% This is a comment block
% that spans multiple lines
% and should be foldable as a unit
% when the folding system detects it

\section{Discussion}
Final section with some discussion text.

\subsection{Limitations}
Discussion of limitations.

\subsection{Future Work}
Ideas for future work.

\section{Conclusion}
The conclusion wraps up the document.

\end{document}